%\VignetteEngine{knitr::knitr}
%\VignetteIndexEntry{Examples in statistical ecology}
%\VignetteDepends{tramME, gamm4, gamlss.dist, survival, mgcv, glmmTMB, xtable, multcomp}

\documentclass[11pt]{article}\usepackage[]{graphicx}\usepackage{xcolor}
% maxwidth is the original width if it is less than linewidth
% otherwise use linewidth (to make sure the graphics do not exceed the margin)
\makeatletter
\def\maxwidth{ %
  \ifdim\Gin@nat@width>\linewidth
    \linewidth
  \else
    \Gin@nat@width
  \fi
}
\makeatother

\definecolor{fgcolor}{rgb}{0.345, 0.345, 0.345}
\newcommand{\hlnum}[1]{\textcolor[rgb]{0.686,0.059,0.569}{#1}}%
\newcommand{\hlstr}[1]{\textcolor[rgb]{0.192,0.494,0.8}{#1}}%
\newcommand{\hlcom}[1]{\textcolor[rgb]{0.678,0.584,0.686}{\textit{#1}}}%
\newcommand{\hlopt}[1]{\textcolor[rgb]{0,0,0}{#1}}%
\newcommand{\hlstd}[1]{\textcolor[rgb]{0.345,0.345,0.345}{#1}}%
\newcommand{\hlkwa}[1]{\textcolor[rgb]{0.161,0.373,0.58}{\textbf{#1}}}%
\newcommand{\hlkwb}[1]{\textcolor[rgb]{0.69,0.353,0.396}{#1}}%
\newcommand{\hlkwc}[1]{\textcolor[rgb]{0.333,0.667,0.333}{#1}}%
\newcommand{\hlkwd}[1]{\textcolor[rgb]{0.737,0.353,0.396}{\textbf{#1}}}%
\let\hlipl\hlkwb

\usepackage{framed}
\makeatletter
\newenvironment{kframe}{%
 \def\at@end@of@kframe{}%
 \ifinner\ifhmode%
  \def\at@end@of@kframe{\end{minipage}}%
  \begin{minipage}{\columnwidth}%
 \fi\fi%
 \def\FrameCommand##1{\hskip\@totalleftmargin \hskip-\fboxsep
 \colorbox{shadecolor}{##1}\hskip-\fboxsep
     % There is no \\@totalrightmargin, so:
     \hskip-\linewidth \hskip-\@totalleftmargin \hskip\columnwidth}%
 \MakeFramed {\advance\hsize-\width
   \@totalleftmargin\z@ \linewidth\hsize
   \@setminipage}}%
 {\par\unskip\endMakeFramed%
 \at@end@of@kframe}
\makeatother

\definecolor{shadecolor}{rgb}{.97, .97, .97}
\definecolor{messagecolor}{rgb}{0, 0, 0}
\definecolor{warningcolor}{rgb}{1, 0, 1}
\definecolor{errorcolor}{rgb}{1, 0, 0}
\newenvironment{knitrout}{}{} % an empty environment to be redefined in TeX

\usepackage{alltt}
\usepackage{graphicx}
\usepackage[margin=0.8in]{geometry}
\usepackage[margin=0.5in, font=small]{caption}
\usepackage{amsfonts,amstext,amsmath,amssymb,amsthm}
\usepackage[utf8]{inputenc}
\usepackage[comma,authoryear]{natbib}
\usepackage{xcolor}
\usepackage[colorlinks=true, urlcolor=black, linkcolor=blue, citecolor=blue]{hyperref}
\usepackage{booktabs}
\usepackage{doi}


\renewcommand{\tt}[1]{\texttt{#1}}
\newcommand{\todo}[1]{\textbf{\color{red} TODO: #1}}
\newcommand{\note}[1]{\textbf{\color{orange} NOTE: #1}}
\newcommand{\code}[1]{\texttt{#1}}
\newcommand{\pkg}[1]{\texttt{#1}}
\newcommand{\CRANpkg}[1]{\href{https://CRAN.R-project.org/package=#1}{\pkg{#1}}}%

% Math notation
\newcommand{\IP}{{\mathbb{P}}}
\newcommand{\IR}{{\mathbb{R}}}
% \newcommand{\U}{{\mathbf{U}}}
\newcommand{\bcU}{{\boldsymbol{\mathcal{U}}}}
\newcommand{\bfu}{{\mathbf{u}}}
% \newcommand{\X}{{\mathbf{X}}}
\newcommand{\X}{{\boldsymbol{X}}}
% \newcommand{\x}{{\mathbf{x}}}
\newcommand{\x}{{\boldsymbol{x}}}
\newcommand{\bcX}{{\boldsymbol{\mathcal{X}}}}
\newcommand{\Y}{{\boldsymbol{Y}}}
\newcommand{\bcY}{{\boldsymbol{\mathcal{Y}}}}
\newcommand{\y}{{\boldsymbol{y}}}
\newcommand{\bbeta}{{\boldsymbol{\beta}}}
\newcommand{\bvartheta}{{\boldsymbol{\vartheta}}}
\newcommand{\bgamma}{{\boldsymbol{\gamma}}}
\newcommand{\bGamma}{{\boldsymbol{\Gamma}}}
\newcommand{\bSigma}{{\boldsymbol{\Sigma}}{}}
\newcommand{\vect}[1]{\boldsymbol{#1}}
\newcommand{\cN}{{\mathcal{N}}}
\newcommand{\cL}{{\mathcal{L}}}
\newcommand{\0}{{\mathbf{0}}}
\renewcommand{\a}{{\mathbf{a}}}
\newcommand{\T}{^{\top}}
\renewcommand{\d}{\mathsf{\,d}}

\usepackage{accents}
\newcommand{\ubar}[1]{\underaccent{\bar}{#1}}

%%%%%% Column header formatting in xtables
%% https://stackoverflow.com/a/33237779/10826854
\usepackage{array}
\newcolumntype{R}[1]{>{\raggedleft\let\newline\\\arraybackslash\hspace{0pt}}p{#1}}

\setlength{\parskip}{0.5em}
\renewcommand{\baselinestretch}{1.2}

% \newcommand{\address}[1]{
%   \addvspace{\baselineskip}\noindent
%   \textbf{Affiliations:} \\
%   \emph{#1}}
% \newcommand{\email}[1]{\href{mailto:#1}{\normalfont\texttt{#1}}}



\title{Example Analyses in Statistical Ecology}
\author{B\'alint Tam\'asi \\ {\small \url{balint.tamasi@uzh.ch}}}

\date{3/3/2022}
\IfFileExists{upquote.sty}{\usepackage{upquote}}{}
\begin{document}

\maketitle

\abstract{
  This vignette serves as an online appendix for the manuscript
  \citet{Tamasi_2022}. It presents three example analyses that
  use mixed-effects additive transformation models to reanalyze
  datasets from recently published studies in the field of ecology.
}


\section{\emph{E. coli} concentrations in streams
  with different grazing periods}\label{sec:ecoli}

\citet{Hulvey_2021} compare the concentration levels
of \emph{Escherichia coli} bacteria (most probable number, MPN)
in streams under three different rotational grazing regimes.
In the additive mixed model specifications they estimated,
within-year variability was modeled,
as functions of the day of year (DOY),
with cubic regression splines
and between-year and location-level variability was captured by
random intercepts of pasture-specific year effects and separate stream effects.
Note that although the cyclic version of the cubic regression splines
(\texttt{bs = 'cc'} in \pkg{mgcv} and \pkg{tramME})
would be more appropriate for modeling the within-year trend,
the original article used \texttt{bs = 'cr'}
and hence we also stick with this basis in our reanalysis.



As a first step,
we replicate the results of all model variants
that they investigated in the original article
with the \textsf{R} package \pkg{gamm4} \citep{gamm4}.
Next, we reproduce the results with additive transformation models
assuming conditional normality
and, finally, relax the distributional assumption and
evaluate how the model fits change.
As Table~\ref{tbl:ecoli-res} shows,
we managed to reproduce the \pkg{gamm4} results with \pkg{tramME}.
Moreover, relaxing the distributional assumption of the
normal linear model resulted in stronger model fits
in terms of log-likelihood values.

\begin{knitrout}\small
\definecolor{shadecolor}{rgb}{0.969, 0.969, 0.969}\color{fgcolor}\begin{kframe}
\begin{alltt}
\hlstd{R> }\hlcom{## specifications w/o random effects}
\hlstd{R> }\hlstd{mf} \hlkwb{<-} \hlkwd{c}\hlstd{(}\hlkwd{log10}\hlstd{(ecoli_MPN)} \hlopt{~} \hlstd{treatment} \hlopt{+} \hlstd{cattle} \hlopt{+}
\hlstd{+  }          \hlkwd{s}\hlstd{(DOY,} \hlkwc{bs} \hlstd{=} \hlstr{'cr'}\hlstd{,} \hlkwc{by} \hlstd{= treatment),}
\hlstd{+  }        \hlkwd{log10}\hlstd{(ecoli_MPN)} \hlopt{~} \hlstd{treatment} \hlopt{+} \hlstd{cattle} \hlopt{+} \hlkwd{s}\hlstd{(DOY,} \hlkwc{bs} \hlstd{=} \hlstr{'cr'}\hlstd{),}
\hlstd{+  }        \hlkwd{log10}\hlstd{(ecoli_MPN)} \hlopt{~} \hlstd{treatment} \hlopt{+} \hlkwd{s}\hlstd{(DOY,} \hlkwc{bs} \hlstd{=} \hlstr{'cr'}\hlstd{,} \hlkwc{by} \hlstd{= treatment),}
\hlstd{+  }        \hlkwd{log10}\hlstd{(ecoli_MPN)} \hlopt{~} \hlstd{cattle} \hlopt{+} \hlkwd{s}\hlstd{(DOY,} \hlkwc{bs} \hlstd{=} \hlstr{'cr'}\hlstd{),}
\hlstd{+  }        \hlkwd{log10}\hlstd{(ecoli_MPN)} \hlopt{~} \hlstd{treatment} \hlopt{+} \hlkwd{s}\hlstd{(DOY,} \hlkwc{bs} \hlstd{=} \hlstr{'cr'}\hlstd{),}
\hlstd{+  }        \hlkwd{log10}\hlstd{(ecoli_MPN)} \hlopt{~} \hlkwd{s}\hlstd{(DOY,} \hlkwc{bs} \hlstd{=} \hlstr{'cr'}\hlstd{))}
\hlstd{R> }\hlkwd{names}\hlstd{(mf)} \hlkwb{<-} \hlkwd{paste}\hlstd{(}\hlstr{"Model"}\hlstd{,} \hlkwd{c}\hlstd{(}\hlnum{1}\hlopt{:}\hlnum{5}\hlstd{,} \hlstr{"Null"}\hlstd{))}
\hlstd{R> }\hlstd{ecoli_res} \hlkwb{<-} \hlkwd{data.frame}\hlstd{(}\hlkwd{matrix}\hlstd{(}\hlnum{NA}\hlstd{,} \hlkwc{nrow} \hlstd{=} \hlkwd{length}\hlstd{(mf),} \hlkwc{ncol} \hlstd{=} \hlnum{3}\hlstd{))}
\hlstd{R> }\hlkwd{colnames}\hlstd{(ecoli_res)} \hlkwb{<-} \hlkwd{c}\hlstd{(}\hlstr{"gamm"}\hlstd{,} \hlstr{"LmME"}\hlstd{,} \hlstr{"BoxCoxME"}\hlstd{)}
\hlstd{R> }\hlkwd{rownames}\hlstd{(ecoli_res)} \hlkwb{<-} \hlkwd{names}\hlstd{(mf)}
\hlstd{R> }\hlkwa{for} \hlstd{(i} \hlkwa{in} \hlkwd{seq_along}\hlstd{(mf)) \{}
\hlstd{+  }  \hlstd{m_gamm} \hlkwb{<-} \hlkwd{gamm4}\hlstd{(mf[[i]],} \hlkwc{data} \hlstd{= ecoli,}
\hlstd{+  }                  \hlkwc{random} \hlstd{=} \hlopt{~} \hlstd{(}\hlnum{1} \hlopt{|} \hlstd{year}\hlopt{:}\hlstd{stream}\hlopt{:}\hlstd{pasture)} \hlopt{+} \hlstd{(}\hlnum{1} \hlopt{|} \hlstd{stream),}
\hlstd{+  }                  \hlkwc{REML} \hlstd{=} \hlnum{FALSE}\hlstd{)}
\hlstd{+  }  \hlstd{ecoli_res}\hlopt{$}\hlstd{gamm[i]} \hlkwb{<-} \hlkwd{logLik}\hlstd{(m_gamm}\hlopt{$}\hlstd{mer)}
\hlstd{+  }  \hlstd{mf2} \hlkwb{<-} \hlkwd{update}\hlstd{(mf[[i]], .} \hlopt{~} \hlstd{.} \hlopt{+} \hlstd{(}\hlnum{1} \hlopt{|} \hlstd{year}\hlopt{:}\hlstd{stream}\hlopt{:}\hlstd{pasture)} \hlopt{+} \hlstd{(}\hlnum{1} \hlopt{|} \hlstd{stream))}
\hlstd{+  }  \hlstd{m_LmME} \hlkwb{<-} \hlkwd{LmME}\hlstd{(mf2,} \hlkwc{data} \hlstd{= ecoli)}
\hlstd{+  }  \hlkwa{if} \hlstd{(m_LmME}\hlopt{$}\hlstd{opt}\hlopt{$}\hlstd{convergence} \hlopt{==} \hlnum{0}\hlstd{) ecoli_res}\hlopt{$}\hlstd{LmME[i]} \hlkwb{<-} \hlkwd{logLik}\hlstd{(m_LmME)}
\hlstd{+  }  \hlstd{m_BCME} \hlkwb{<-} \hlkwd{BoxCoxME}\hlstd{(mf2,} \hlkwc{data} \hlstd{= ecoli)}
\hlstd{+  }  \hlkwa{if} \hlstd{(m_BCME}\hlopt{$}\hlstd{opt}\hlopt{$}\hlstd{convergence} \hlopt{==} \hlnum{0}\hlstd{) ecoli_res}\hlopt{$}\hlstd{BoxCoxME[i]} \hlkwb{<-} \hlkwd{logLik}\hlstd{(m_BCME)}
\hlstd{+  }\hlstd{\}}
\end{alltt}
\end{kframe}
\end{knitrout}

\begin{table}[!ht]
  \centering
  \caption{Log-likelihood values of the fitted models
    presented by \protect\citet[\emph{GAMM}]{Hulvey_2021}, replicated as
    mixed-effects additive transformation models assuming conditional normality
    (\emph{Additive normal transformation model}) and extended
    as flexible (non-normal) mixed-effects additive transformation models
    (\emph{Additive non-normal transformation model}).}
  \label{tbl:ecoli-res}

% latex table generated in R 4.1.2 by xtable 1.8-4 package
% Mon Mar  7 15:39:09 2022
\begin{tabular}{lR{2cm}R{4cm}R{4cm}}
  \toprule
 & \multicolumn{1}{m{2cm}}{\centering GAMM} & \multicolumn{1}{m{4cm}}{\centering Additive normal transformation model} & \multicolumn{1}{m{4cm}}{\centering Additive non-normal transformation model} \\ 
  \midrule
Model 1 & -339.23 & -339.23 & -320.94 \\ 
  Model 2 & -343.66 & -343.66 & -324.54 \\ 
  Model 3 & -368.33 & -368.33 & -349.10 \\ 
  Model 4 & -347.70 & -347.70 & -328.25 \\ 
  Model 5 & -367.15 & -367.15 & -347.27 \\ 
  Model Null & -373.76 & -373.76 & -353.50 \\ 
   \bottomrule
\end{tabular}


\end{table}

Let us focus on the most complicated specification, Model~1,
%
\begin{knitrout}\small
\definecolor{shadecolor}{rgb}{0.969, 0.969, 0.969}\color{fgcolor}\begin{kframe}
\begin{alltt}
\hlstd{R> }\hlkwd{update}\hlstd{(mf[[}\hlnum{1}\hlstd{]], .} \hlopt{~} \hlstd{.} \hlopt{+} \hlstd{(}\hlnum{1} \hlopt{|} \hlstd{year}\hlopt{:}\hlstd{stream}\hlopt{:}\hlstd{pasture)} \hlopt{+} \hlstd{(}\hlnum{1} \hlopt{|} \hlstd{stream))}
\end{alltt}
\begin{verbatim}
log10(ecoli_MPN) ~ treatment + cattle + s(DOY, bs = "cr", by = treatment) + 
    (1 | year:stream:pasture) + (1 | stream)
\end{verbatim}
\end{kframe}
\end{knitrout}
%
\noindent
and compare the effect estimates from the normal model to
its non-parametric counterpart.
But first, notice that by changing the transformation from
$h(y) = \vartheta_{0} + \vartheta_{1}y$ to
$h(y) = \a(y)\T\bvartheta$,
we change the scale on which the coefficients
and the smooth terms are interpreted.
In the normal additive mixed model,
the coefficient of a fixed effect captures
the change in the expectation of the outcome when increasing
the respective predictor by one unit
(keeping everything else unchanged).
In the non-normal transformation model with
$\Phi$ as the inverse link, the coefficients
capture similar effects but on a latent scale
defined by the transformation $h(Y)$.

To cast the effect estimates from the two models to a common scale,
we can calculate the \emph{probabilistic indices} \citep[PI,][]{Thas_2012}.
To simplify the notation,
without loss of generality,
we will now focus on the simple, fixed effects-only case:
%
\begin{align*}
  \IP(Y\leq y \mid \X = \x) &= \Phi\left(h(y) - \x\T\bbeta \right)
\end{align*}
%
The PI is the probability that
one outcome ($Y^{\star}$) is larger than the other ($Y$),
given the same covariate values ($\X$) except for one,
which is larger with one unit ($\X^{\star}$)
%
\begin{align*}
  \IP\left(Y < Y^{\star} \mid \X, \X^{\star}\right)
  &= \IP\left(h(Y) < h(Y^{\star}) \mid \X, \X^{\star}\right) \\
  &= \IP\left(\left.\frac{h(Y) - h(Y^{\star}) + \beta}{\sqrt{2}} < \frac{\beta}{\sqrt{2}}
    \right| \X, \X^{\star}\right) \\
  &= \Phi\left(\frac{\beta}{\sqrt{2}}\right),
\end{align*}
%
where $\beta$ is the coefficient
of the covariate that is different with one unit.
The third line is true, because $h(Y)$ and $h(Y^{\star})$
are independent, normally distributed random variables,
with unit variance and a mean difference of $\beta$.
Notice that the PI does not depend on the transformation function.
When random effects are present in the model,
the PI is conditional on the cluster.

By casting the effect estimates to the probability scale,
Figure~\ref{fig:ecoli-nonnorm} compares the smooth terms from
the normal and non-normal versions of Model~1,
while the first two blocks of Table~\ref{tbl:ecoli-fes} the fixed effects estimates.
The results are very close to each other,
which suggests that the original log-normal model is actually appropriate.
As a built-in visual normality check,
we can compare the fitted transformation functions
of the normal and non-normal transformation models.
The linear function corresponds
to normal conditional distribution in Figure~\ref{fig:ecoli-trafo}.
This result further confirms the appropriateness
of the normal additive model in this specific example.





\begin{figure}[!ht]
  \centering

\begin{knitrout}\small
\definecolor{shadecolor}{rgb}{0.969, 0.969, 0.969}\color{fgcolor}
\includegraphics[width=.9\linewidth]{figure/plot-ecoli-m1-1} 
\end{knitrout}

\caption{The comparison of the smooth terms from the normal
  and non-normal (probit link) mixed-effects additive transformation models
  (specification Model~1).
}\label{fig:ecoli-nonnorm}
\end{figure}

\begin{figure}[!ht]
\centering

\begin{knitrout}\small
\definecolor{shadecolor}{rgb}{0.969, 0.969, 0.969}\color{fgcolor}
\includegraphics[width=0.5\linewidth]{figure/plot-ecoli-trafo-1} 
\end{knitrout}

\caption{Baseline transformation functions
  from the normal and non-normal mixed-effects
  additive transformation models.}\label{fig:ecoli-trafo}
\end{figure}


The outcome variable (MPN per 100~ml)
was measured with the Quanti-Tray System,
which can detect \emph{E. coli} concentrations
up to a maximum of 2,419.6~MPN without dilution.
This means that there is an effective upper detection limit
on the outcome,
i.e., the 25 observations with
the value of 2,419.6 are \emph{right censored}.
The authors of the original article mention this fact,
but they do not take into account in the subsequent analyses.
Because censoring can be easily handled in \pkg{tramME},
we will rerun the model taking the upper limit into account.
%
\begin{knitrout}\small
\definecolor{shadecolor}{rgb}{0.969, 0.969, 0.969}\color{fgcolor}\begin{kframe}
\begin{alltt}
\hlstd{R> }\hlstd{fm1c} \hlkwb{<-} \hlkwd{update}\hlstd{(fm1,} \hlkwd{Surv}\hlstd{(}\hlkwd{log10}\hlstd{(ecoli_MPN),} \hlkwc{event} \hlstd{= ecoli_MPN} \hlopt{<} \hlnum{2419.6}\hlstd{)} \hlopt{~} \hlstd{.)}
\hlstd{R> }\hlstd{ecoli_m1_cens} \hlkwb{<-} \hlkwd{BoxCoxME}\hlstd{(fm1c,} \hlkwc{data} \hlstd{= ecoli)}
\hlstd{R> }\hlkwd{summary}\hlstd{(ecoli_m1_cens)}
\end{alltt}
\begin{verbatim}

Non-Normal (Box-Cox-Type) Linear Additive Mixed-Effects Regression Model

	Formula: Surv(log10(ecoli_MPN), event = ecoli_MPN < 2419.6) ~ treatment + 
    cattle + s(DOY, bs = "cr", by = treatment) + (1 | year:stream:pasture) + 
    (1 | stream)

	Fitted to dataset ecoli  

	Fixed effects parameters:
	=========================

                Estimate Std. Error z value Pr(>|z|)    
treatmentmedium   -0.680      0.230   -2.95   0.0032 ** 
treatmentshort    -0.772      0.317   -2.44   0.0148 *  
cattlePresent      1.108      0.149    7.42  1.2e-13 ***
---
Signif. codes:  0 '***' 0.001 '**' 0.01 '*' 0.05 '.' 0.1 ' ' 1

	Smooth shift terms:
	===================

                        edf
s(DOY):treatmentlong   4.38
s(DOY):treatmentmedium 4.55
s(DOY):treatmentshort  4.30

	Random effects:
	===============

Grouping factor: year:stream:pasture (32 levels)
Standard deviation:
(Intercept) 
      0.431 

Grouping factor: stream (12 levels)
Standard deviation:
(Intercept) 
   0.000204 


	Log-likelihood: -358 (npar = 18)
\end{verbatim}
\end{kframe}
\end{knitrout}
%
The fitted non-linear terms are compared to the original
(normal linear) estimates in Figure~\ref{fig:ecoli-cens}
and the fixed effects are presented in the third block of Table~\ref{tbl:ecoli-fes}.

\begin{table}[!ht]
  \centering
  \caption{
    Estimates of the parametric fixed-effects
    terms on the \emph{probability scale}
    (PI:~probabilistic index) from the normal, non-normal
    and non-normal (with censoring taken into account) models,
    respectively.
  }\label{tbl:ecoli-fes}
% latex table generated in R 4.1.2 by xtable 1.8-4 package
% Mon Mar  7 15:39:11 2022
\begin{tabular}{lrrrrrr}
  \toprule
  & \multicolumn{2}{c}{Normal} & \multicolumn{2}{c}{Non-normal} & \multicolumn{2}{c}{Non-normal, censored} \\
 \cmidrule(lr){2-3} \cmidrule(lr){4-5} \cmidrule(lr){6-7} & PI & 95\% CI  & PI & 95\% CI  & PI & 95\% CI  \\
 \midrule
treatment = medium & 0.32 & 0.22---0.44 & 0.33 & 0.22---0.45 & 0.32 & 0.21---0.44 \\ 
  treatment = short & 0.29 & 0.16---0.45 & 0.30 & 0.17---0.46 & 0.29 & 0.16---0.46 \\ 
  cattle = present & 0.79 & 0.72---0.84 & 0.78 & 0.72---0.84 & 0.78 & 0.72---0.84 \\ 
   \bottomrule
\end{tabular}


\end{table}


\begin{figure}[!ht]
  \centering

\begin{knitrout}\small
\definecolor{shadecolor}{rgb}{0.969, 0.969, 0.969}\color{fgcolor}
\includegraphics[width=.9\linewidth]{figure/plot-ecoli-cens-1} 
\end{knitrout}

\caption{
  The comparison of the smooth terms
  from the original model (normal linear)
  and the non-normal (probit link) extension
  where censoring is also taken into account.
}\label{fig:ecoli-cens}
\end{figure}

Because the transformation model approximates
the conditional distribution of the outcome,
in theory,
we do not even have to take the base 10 logarithm
of the \emph{Ecoli} most probable numbers (MPN)
on the left-hand side of the model formula.
\pkg{tramME} should be able to approximate the
\emph{most likely transformation}.
%
\begin{knitrout}\small
\definecolor{shadecolor}{rgb}{0.969, 0.969, 0.969}\color{fgcolor}\begin{kframe}
\begin{alltt}
\hlstd{R> }\hlstd{f_nontr} \hlkwb{<-} \hlkwd{update}\hlstd{(fm1,} \hlkwd{Surv}\hlstd{(ecoli_MPN,} \hlkwc{event} \hlstd{= ecoli_MPN} \hlopt{<} \hlnum{2419.6}\hlstd{)} \hlopt{~} \hlstd{.)}
\hlstd{R> }\hlstd{ecoli_nontr} \hlkwb{<-} \hlkwd{BoxCoxME}\hlstd{(f_nontr,} \hlkwc{data} \hlstd{= ecoli,} \hlkwc{log_first} \hlstd{=} \hlnum{TRUE}\hlstd{)}
\hlstd{R> }\hlkwd{summary}\hlstd{(ecoli_nontr)}
\end{alltt}
\begin{verbatim}

Non-Normal (Box-Cox-Type) Linear Additive Mixed-Effects Regression Model

	Formula: Surv(ecoli_MPN, event = ecoli_MPN < 2419.6) ~ treatment + cattle + 
    s(DOY, bs = "cr", by = treatment) + (1 | year:stream:pasture) + 
    (1 | stream)

	Fitted to dataset ecoli  

	Fixed effects parameters:
	=========================

                Estimate Std. Error z value Pr(>|z|)    
treatmentmedium   -0.680      0.230   -2.95   0.0032 ** 
treatmentshort    -0.772      0.317   -2.44   0.0148 *  
cattlePresent      1.108      0.149    7.42  1.2e-13 ***
---
Signif. codes:  0 '***' 0.001 '**' 0.01 '*' 0.05 '.' 0.1 ' ' 1

	Smooth shift terms:
	===================

                        edf
s(DOY):treatmentlong   4.38
s(DOY):treatmentmedium 4.55
s(DOY):treatmentshort  4.30

	Random effects:
	===============

Grouping factor: year:stream:pasture (32 levels)
Standard deviation:
(Intercept) 
      0.431 

Grouping factor: stream (12 levels)
Standard deviation:
(Intercept) 
   0.000394 


	Log-likelihood: -2027 (npar = 18)
\end{verbatim}
\end{kframe}
\end{knitrout}
%
Notice that we set \texttt{log\_first = TRUE} in the function call,
to take the natural logarithm of the outcome
before setting up the Bernstein bases.
This usually helps the approximation in the case of positive right-skewed outcomes.
With this, we basically estimate the same model as the original,
but with the natural logarithm instead of base-ten.
Because of this, the log-likelihood values are different,
but the fixed effects and variance components parameter estimates,
as well as the smooth terms are essentially the same as
in the case of the model \texttt{ecoli\_m1\_cens}.

In summary,
after bringing the estimates to the same scale,
the results of the additive mixed effects model
did not change much in this specific example
by switching to the transformation model approach.
The originally applied base 10 logarithm falls very close
to the fitted ``most likely transformation'',
i.e., taking the logarithm of the outcome was sufficient
to achieve (close) conditional normality.
This could be verified through comparing
the baseline transformation functions of the normal and non-normal models,
which can also serve as a visual check on conditional normality.
Moreover, the number of censored outcomes
was relatively small in the sample,
so taking the censoring properly into account
did not result in large differences, either.
However, as the example demonstrated,
transformation models are flexible enough
to accommodate these properties
of the response of interest
(non-normality and censoring) automatically,
without the need to apply ad hoc transformations
or to implement new estimation procedures.


\section{Sea urchin removal experiment}

\citet{Andrew_1993} analyzed the percentage cover
of filamentous algae
under four sea urchin removal treatments (Control/33\%/66\%/Removal).
The algae colonization was measured
on five quadrants located on several larger patches,
so there is a clear grouped structure in the data.
\citet{Douma_2019} reanalyzed the data as a demonstration for the usage
of mixed-effects models for zero-inflated beta regression models.
Here we fit mixed-effects transformation models to the data,
and compare the results to zero-inflated mixed-model estimates
obtained from \texttt{glmmTMB} \citep{Brooks_2017}.
Figure~\ref{fig:algae-data} presents the empirical cumulative distribution functions
of the outcome under the four treatments.
Note the large number of zeros, especially in the control group.



\begin{figure}[!ht]
  \centering
\begin{knitrout}\small
\definecolor{shadecolor}{rgb}{0.969, 0.969, 0.969}\color{fgcolor}
\includegraphics[width=.6\textwidth]{figure/plot-algae-treatment-1} 
\end{knitrout}
\caption{Empirical CDFs of the algae cover proportions under the four treatments.}\label{fig:algae-data}
\end{figure}

First we fit a zero-inflated beta regression model with random intercepts
for the patches. The probability of observing zero values depends is allowed to
vary with the treatment.
%
\begin{knitrout}\small
\definecolor{shadecolor}{rgb}{0.969, 0.969, 0.969}\color{fgcolor}\begin{kframe}
\begin{alltt}
\hlstd{R> }\hlstd{urchin_zib} \hlkwb{<-} \hlkwd{glmmTMB}\hlstd{(pALGAE} \hlopt{~} \hlstd{TREAT} \hlopt{+} \hlstd{(}\hlnum{1} \hlopt{|} \hlstd{PATCH),} \hlkwc{ziformula} \hlstd{=} \hlopt{~} \hlstd{TREAT,}
\hlstd{+  }                      \hlkwc{data} \hlstd{= andrew,} \hlkwc{family} \hlstd{=} \hlkwd{beta_family}\hlstd{())}
\hlstd{R> }\hlkwd{summary}\hlstd{(urchin_zib)}
\end{alltt}
\begin{verbatim}
 Family: beta  ( logit )
Formula:          pALGAE ~ TREAT + (1 | PATCH)
Zero inflation:          ~TREAT
Data: andrew

     AIC      BIC   logLik deviance df.resid 
    87.2    111.0    -33.6     67.2       70 

Random effects:

Conditional model:
 Groups Name        Variance Std.Dev.
 PATCH  (Intercept) 0.124    0.352   
Number of obs: 80, groups:  PATCH, 16

Dispersion parameter for beta family (): 4.06 

Conditional model:
             Estimate Std. Error z value Pr(>|z|)    
(Intercept)    -2.060      0.530   -3.89   0.0001 ***
TREAT0.33       1.280      0.614    2.08   0.0372 *  
TREAT0.66       1.374      0.602    2.28   0.0223 *  
TREATremoval    1.783      0.585    3.05   0.0023 ** 
---
Signif. codes:  0 '***' 0.001 '**' 0.01 '*' 0.05 '.' 0.1 ' ' 1

Zero-inflation model:
             Estimate Std. Error z value Pr(>|z|)    
(Intercept)     1.099      0.516    2.13  0.03338 *  
TREAT0.33      -1.299      0.685   -1.90  0.05772 .  
TREAT0.66      -1.504      0.689   -2.18  0.02908 *  
TREATremoval   -2.833      0.812   -3.49  0.00048 ***
---
Signif. codes:  0 '***' 0.001 '**' 0.01 '*' 0.05 '.' 0.1 ' ' 1
\end{verbatim}
\end{kframe}
\end{knitrout}

As an alternative to the traditional beta regression approach,
we estimate a mixed-effects continuous outcome logistic regression.
%
\begin{knitrout}\small
\definecolor{shadecolor}{rgb}{0.969, 0.969, 0.969}\color{fgcolor}\begin{kframe}
\begin{alltt}
\hlstd{R> }\hlstd{urchin_tram} \hlkwb{<-} \hlkwd{ColrME}\hlstd{(}
\hlstd{+  }  \hlkwd{Surv}\hlstd{(pALGAE, pALGAE} \hlopt{>} \hlnum{0}\hlstd{,} \hlkwc{type} \hlstd{=} \hlstr{"left"}\hlstd{)} \hlopt{~} \hlstd{TREAT} \hlopt{+} \hlstd{(}\hlnum{1} \hlopt{|} \hlstd{PATCH),}
\hlstd{+  }  \hlkwc{bounds} \hlstd{=} \hlkwd{c}\hlstd{(}\hlopt{-}\hlnum{0.1}\hlstd{,} \hlnum{1}\hlstd{),} \hlkwc{support} \hlstd{=} \hlkwd{c}\hlstd{(}\hlopt{-}\hlnum{0.1}\hlstd{,} \hlnum{1}\hlstd{),} \hlkwc{data} \hlstd{= andrew,}
\hlstd{+  }  \hlkwc{order} \hlstd{=} \hlnum{6}\hlstd{)}
\hlstd{R> }\hlkwd{summary}\hlstd{(urchin_tram)}
\end{alltt}
\begin{verbatim}

Mixed-Effects Continuous Outcome Logistic Regression Model

	Formula: Surv(pALGAE, pALGAE > 0, type = "left") ~ TREAT + (1 | PATCH)

	Fitted to dataset andrew  

	Fixed effects parameters:
	=========================

             Estimate Std. Error z value Pr(>|z|)   
TREAT0.33       -2.04       1.31   -1.56   0.1178   
TREAT0.66       -2.49       1.31   -1.90   0.0571 . 
TREATremoval    -4.10       1.34   -3.06   0.0022 **
---
Signif. codes:  0 '***' 0.001 '**' 0.01 '*' 0.05 '.' 0.1 ' ' 1

	Random effects:
	===============

Grouping factor: PATCH (16 levels)
Standard deviation:
(Intercept) 
       1.48 


	Log-likelihood: -26.3 (npar = 11)
\end{verbatim}
\end{kframe}
\end{knitrout}
%
To allow for a jump in the CDF of the outcome,
we expand its bound and treat the zero observations as left-censored.
This way, we can place a point mass on zero,
i.e., introduce a jump at 0
(see Figure~\ref{fig:cdf-jump}).

\begin{figure}[!ht]
  \centering

\begin{knitrout}\small
\definecolor{shadecolor}{rgb}{0.969, 0.969, 0.969}\color{fgcolor}
\includegraphics[width=.8\textwidth]{figure/pointmass-plot-1} 
\end{knitrout}

\caption{Visual demonstration of how a discrete jump
  is introduced in the CDF by extending the support and
  treating the edge cases as censored.}\label{fig:cdf-jump}
\end{figure}

Because the zero-inflated beta model is a mixture of two models,
the interpretation of its results is cumbersome.
It is not clear which parameters,
or combinations of parameters,
one needs to inspect to contrast
the effects of the various treatments.
Moreover, extra steps are needed to calculate the marginal effects
of the covariates.
In contrast,
the mixed-effects transformation model
only contains a single set of fixed effects parameters
and their interpretation is straightforward:
For example,
the odds of observing higher proportions of algae cover
under the 33\% removal treatment is about
$\exp(-\widehat{\beta}_{0.33}) = $
7.71
times higher compared to the control group.

To assess the fits of the two models we can marginalize
the conditional distributions by integrating over the random effects numerically,
and compare against the ECDFs.
As Figure~\ref{fig:algae-mcdf1} shows, both model overestimate the dispersion
in the control group.



\begin{figure}[!ht]
\centering

\begin{knitrout}\small
\definecolor{shadecolor}{rgb}{0.969, 0.969, 0.969}\color{fgcolor}
\includegraphics[width=.9\linewidth]{figure/plot-algae-mCDF1-1} 
\end{knitrout}

\caption{Fitted marginal distributions of algae cover proportion from
  the zero-inflated beta regression and the mixed-effects
  transformation model, respectively.
  The step functions show the empirical cumulative distribution
  functions in the four treatment groups.}\label{fig:algae-mcdf1}
\end{figure}

Systematic differences in the outcome variability
in the treatment groups occur in many situations \citep{Douma_2019}.
By modeling the dispersion separately, we can incorporate such differences
in the beta regression model.
%
\begin{knitrout}\small
\definecolor{shadecolor}{rgb}{0.969, 0.969, 0.969}\color{fgcolor}\begin{kframe}
\begin{alltt}
\hlstd{R> }\hlstd{urchin_zib_disp} \hlkwb{<-} \hlkwd{glmmTMB}\hlstd{(pALGAE} \hlopt{~} \hlstd{TREAT} \hlopt{+} \hlstd{(}\hlnum{1} \hlopt{|} \hlstd{PATCH),}
\hlstd{+  }                           \hlkwc{ziformula} \hlstd{=} \hlopt{~} \hlstd{TREAT,} \hlkwc{dispformula} \hlstd{=} \hlopt{~} \hlstd{TREAT,}
\hlstd{+  }                           \hlkwc{data} \hlstd{= andrew,} \hlkwc{family} \hlstd{=} \hlkwd{beta_family}\hlstd{())}
\hlstd{R> }\hlkwd{summary}\hlstd{(urchin_zib_disp)}
\end{alltt}
\begin{verbatim}
 Family: beta  ( logit )
Formula:          pALGAE ~ TREAT + (1 | PATCH)
Zero inflation:          ~TREAT
Dispersion:              ~TREAT
Data: andrew

     AIC      BIC   logLik deviance df.resid 
    87.9    118.8    -30.9     61.9       67 

Random effects:

Conditional model:
 Groups Name        Variance Std.Dev.
 PATCH  (Intercept) 0.198    0.445   
Number of obs: 80, groups:  PATCH, 16

Conditional model:
             Estimate Std. Error z value Pr(>|z|)    
(Intercept)    -2.908      0.420   -6.92  4.5e-12 ***
TREAT0.33       2.158      0.587    3.68  0.00023 ***
TREAT0.66       2.213      0.559    3.96  7.6e-05 ***
TREATremoval    2.595      0.523    4.96  7.0e-07 ***
---
Signif. codes:  0 '***' 0.001 '**' 0.01 '*' 0.05 '.' 0.1 ' ' 1

Zero-inflation model:
             Estimate Std. Error z value Pr(>|z|)    
(Intercept)     1.099      0.516    2.13  0.03338 *  
TREAT0.33      -1.299      0.685   -1.90  0.05772 .  
TREAT0.66      -1.504      0.689   -2.18  0.02908 *  
TREATremoval   -2.833      0.812   -3.49  0.00048 ***
---
Signif. codes:  0 '***' 0.001 '**' 0.01 '*' 0.05 '.' 0.1 ' ' 1

Dispersion model:
             Estimate Std. Error z value Pr(>|z|)    
(Intercept)     3.612      0.849    4.26  2.1e-05 ***
TREAT0.33      -2.424      0.925   -2.62   0.0087 ** 
TREAT0.66      -2.279      0.921   -2.47   0.0134 *  
TREATremoval   -2.036      0.870   -2.34   0.0193 *  
---
Signif. codes:  0 '***' 0.001 '**' 0.01 '*' 0.05 '.' 0.1 ' ' 1
\end{verbatim}
\end{kframe}
\end{knitrout}

In the mixed-effects linear transformation model,
we stratify to the treatment group to allow for separate
transformation functions.
%
\begin{knitrout}\small
\definecolor{shadecolor}{rgb}{0.969, 0.969, 0.969}\color{fgcolor}\begin{kframe}
\begin{alltt}
\hlstd{R> }\hlstd{urchin_tram_strat} \hlkwb{<-} \hlkwd{ColrME}\hlstd{(}
\hlstd{+  }  \hlkwd{Surv}\hlstd{(pALGAE, pALGAE} \hlopt{>} \hlnum{0}\hlstd{,} \hlkwc{type} \hlstd{=} \hlstr{"left"}\hlstd{)} \hlopt{|} \hlnum{0} \hlopt{+} \hlstd{TREAT} \hlopt{~} \hlnum{1} \hlopt{+} \hlstd{(}\hlnum{1} \hlopt{|} \hlstd{PATCH),}
\hlstd{+  }  \hlkwc{bounds} \hlstd{=} \hlkwd{c}\hlstd{(}\hlopt{-}\hlnum{0.1}\hlstd{,} \hlnum{1}\hlstd{),} \hlkwc{support} \hlstd{=} \hlkwd{c}\hlstd{(}\hlopt{-}\hlnum{0.1}\hlstd{,} \hlnum{1}\hlstd{),} \hlkwc{data} \hlstd{= andrew,}
\hlstd{+  }  \hlkwc{order} \hlstd{=} \hlnum{6}\hlstd{,} \hlkwc{control} \hlstd{=} \hlkwd{optim_control}\hlstd{(}\hlkwc{iter.max} \hlstd{=} \hlnum{1e3}\hlstd{,} \hlkwc{eval.max} \hlstd{=} \hlnum{1e3}\hlstd{,}
\hlstd{+  }                                     \hlkwc{rel.tol} \hlstd{=} \hlnum{1e-9}\hlstd{))}
\hlstd{R> }\hlkwd{summary}\hlstd{(urchin_tram_strat)}
\end{alltt}
\begin{verbatim}

Stratified Mixed-Effects Continuous Outcome Logistic Regression Model

	Formula: Surv(pALGAE, pALGAE > 0, type = "left") | 0 + TREAT ~ 1 + (1 | 
    PATCH)

	Fitted to dataset andrew  

	Fixed effects parameters:
	=========================

No estimated shift coefficients.

	Random effects:
	===============

Grouping factor: PATCH (16 levels)
Standard deviation:
(Intercept) 
       1.51 


	Log-likelihood: -22.9 (npar = 29)
\end{verbatim}
\end{kframe}
\end{knitrout}



As Figure~\ref{fig:algae-mcdf2} illustrates,
the two models fit the data much better.
However, the cost of this flexibility is
that we cannot reduce the group comparisons
to inference of a small set of parameters anymore.

\begin{figure}[!ht]
\centering

\begin{knitrout}\small
\definecolor{shadecolor}{rgb}{0.969, 0.969, 0.969}\color{fgcolor}
\includegraphics[width=.9\linewidth]{figure/plot-algae-mCDF2-1} 
\end{knitrout}

\caption{Fitted marginal distributions of algae cover proportion from
  the zero-inflated beta regression with dispersion model
  and the stratified mixed-effects transformation model, respectively.
  The step functions show the empirical cumulative distribution
  functions in the four treatment groups.}\label{fig:algae-mcdf2}
\end{figure}

Figures~\ref{fig:algae-mcdf1} and~\ref{fig:algae-mcdf2}
demonstrate the flexibility of the distribution-free
approach of transformation models compared
to the parametric alternative.
This is also reflected in the log-likelihood values (Table~\ref{tbl:ll-algae}).

\begin{table}[!ht]
\centering
\caption{Log-likelihood values of the four model specifications
  for the sea urchin removal experiment.
}\label{tbl:ll-algae}

% latex table generated in R 4.1.2 by xtable 1.8-4 package
% Mon Mar  7 15:39:16 2022
\begin{tabular}{lr}
  \toprule
 & $\log\mathcal{L}$ \\ 
  \midrule
Zero-inflated beta w/o dispersion model & -33.60 \\ 
  Linear transformation model & -26.27 \\ 
  Zero-inflated beta w/ dispersion model & -30.93 \\ 
  Stratified linear transformation model & -22.86 \\ 
   \bottomrule
\end{tabular}


\end{table}

\section{Mosquito control trial}



\citet{Juarez_2021} presented the results of a cluster randomized crossover trial
that assessed the efficacy of Autocidal Gravid Ovitrap (AGO)
as a tool for against the mosquito species \emph{Aedes aegypti}.
The outcome of interest was the number of female mosquitoes
collected on glue boards that were placed either inside or outside
the selected houses in various neighborhoods.
Within-year patterns in mosquito counts as well as coverage of the treatment in
different areas were modeled with non-linear smooths,
while unobserved household and community level effects
were captured by nested random effects.
The original article presented the results
of a conditional Poisson and a negative binomial model.
We reproduce these results with \pkg{gamm4},
and also estimate a mixed-effects additive transformation model
for count data with ``expit'' inverse link function.
Detailed exposition of count transformation models
is given by \citet{Siegfried_Hothorn_2020}.
For this, we will use the following custom-made
\texttt{'CotramME'} model class,
which is currently not part of the \pkg{tramME} package.
%
\begin{knitrout}\small
\definecolor{shadecolor}{rgb}{0.969, 0.969, 0.969}\color{fgcolor}\begin{kframe}
\begin{alltt}
\hlstd{R> }\hlcom{## additive count transformation model}
\hlstd{R> }\hlstd{CotramME} \hlkwb{<-} \hlkwa{function}\hlstd{(}\hlkwc{formula}\hlstd{,} \hlkwc{data}\hlstd{,}
\hlstd{+  }                     \hlkwc{method} \hlstd{=} \hlkwd{c}\hlstd{(}\hlstr{"logit"}\hlstd{,} \hlstr{"cloglog"}\hlstd{,} \hlstr{"loglog"}\hlstd{,} \hlstr{"probit"}\hlstd{),}
\hlstd{+  }                     \hlkwc{log_first} \hlstd{=} \hlnum{TRUE}\hlstd{,} \hlkwc{plus_one} \hlstd{= log_first,} \hlkwc{prob} \hlstd{=} \hlnum{0.9}\hlstd{,}
\hlstd{+  }                     \hlkwc{...}\hlstd{) \{}
\hlstd{+  }  \hlstd{method} \hlkwb{<-} \hlkwd{match.arg}\hlstd{(method)}
\hlstd{+  }  \hlstd{rv} \hlkwb{<-} \hlkwd{all.vars}\hlstd{(formula)[}\hlnum{1}\hlstd{]}
\hlstd{+  }  \hlkwd{stopifnot}\hlstd{(}\hlkwd{is.integer}\hlstd{(data[[rv]]),} \hlkwd{all}\hlstd{(data[[rv]]} \hlopt{>=} \hlnum{0}\hlstd{))}
\hlstd{+  }  \hlstd{data[[rv]]} \hlkwb{<-} \hlstd{data[[rv]]} \hlopt{+} \hlkwd{as.integer}\hlstd{(plus_one)}
\hlstd{+  }  \hlstd{sup} \hlkwb{<-} \hlkwd{c}\hlstd{(}\hlopt{-}\hlnum{0.5} \hlopt{+} \hlstd{log_first,} \hlkwd{quantile}\hlstd{(data[[rv]],} \hlkwc{prob} \hlstd{= prob))}
\hlstd{+  }  \hlstd{bou} \hlkwb{<-} \hlkwd{c}\hlstd{(}\hlopt{-}\hlnum{0.9} \hlopt{+} \hlstd{log_first,} \hlnum{Inf}\hlstd{)}
\hlstd{+  }  \hlstd{data[[rv]]} \hlkwb{<-} \hlkwd{as.Surv}\hlstd{(}\hlkwd{R}\hlstd{(data[[rv]],} \hlkwc{bounds} \hlstd{= bou))}
\hlstd{+  }  \hlstd{fc} \hlkwb{<-} \hlkwd{match.call}\hlstd{()}
\hlstd{+  }  \hlstd{fc[[}\hlnum{1L}\hlstd{]]} \hlkwb{<-} \hlkwd{switch}\hlstd{(method,} \hlkwc{logit} \hlstd{=} \hlkwd{quote}\hlstd{(ColrME),} \hlkwc{cloglog} \hlstd{=} \hlkwd{quote}\hlstd{(CoxphME),}
\hlstd{+  }                     \hlkwc{loglog} \hlstd{=} \hlkwd{quote}\hlstd{(LehmannME),} \hlkwc{probit} \hlstd{=} \hlkwd{quote}\hlstd{(BoxCoxME))}
\hlstd{+  }  \hlstd{fc}\hlopt{$}\hlstd{method} \hlkwb{<-} \hlkwa{NULL}
\hlstd{+  }  \hlstd{fc}\hlopt{$}\hlstd{plus_one} \hlkwb{<-} \hlkwa{NULL}
\hlstd{+  }  \hlstd{fc}\hlopt{$}\hlstd{prob} \hlkwb{<-} \hlkwa{NULL}
\hlstd{+  }  \hlstd{fc}\hlopt{$}\hlstd{log_first} \hlkwb{<-} \hlstd{log_first}
\hlstd{+  }  \hlstd{fc}\hlopt{$}\hlstd{bounds} \hlkwb{<-} \hlstd{bou}
\hlstd{+  }  \hlstd{fc}\hlopt{$}\hlstd{support} \hlkwb{<-} \hlstd{sup}
\hlstd{+  }  \hlstd{fc}\hlopt{$}\hlstd{data} \hlkwb{<-} \hlstd{data}
\hlstd{+  }  \hlstd{out} \hlkwb{<-} \hlkwd{eval}\hlstd{(fc,} \hlkwd{parent.frame}\hlstd{())}
\hlstd{+  }  \hlstd{out}\hlopt{$}\hlstd{call}\hlopt{$}\hlstd{data} \hlkwb{<-} \hlkwd{match.call}\hlstd{()}\hlopt{$}\hlstd{data}
\hlstd{+  }  \hlkwd{class}\hlstd{(out)} \hlkwb{<-} \hlkwd{c}\hlstd{(}\hlstr{"CotramME"}\hlstd{,} \hlkwd{class}\hlstd{(out))}
\hlstd{+  }  \hlstd{out}
\hlstd{+  }\hlstd{\}}
\hlstd{R> }\hlstd{mosquito_tram} \hlkwb{<-} \hlkwd{CotramME}\hlstd{(AEAfemale} \hlopt{~} \hlstd{Year} \hlopt{+} \hlstd{Income}\hlopt{*}\hlstd{Placement}
\hlstd{+  }  \hlopt{+} \hlkwd{s}\hlstd{(Week)} \hlopt{+} \hlkwd{s}\hlstd{(CovRate200)} \hlopt{+} \hlstd{(}\hlnum{1}\hlopt{|}\hlstd{HouseID)}
\hlstd{+  }  \hlopt{+} \hlstd{(}\hlnum{1}\hlopt{|}\hlstd{Community),} \hlkwc{offset} \hlstd{=} \hlopt{-}\hlkwd{log}\hlstd{(daystrapping),} \hlkwc{data} \hlstd{= AGO,}
\hlstd{+  }  \hlkwc{method} \hlstd{=} \hlstr{"logit"}\hlstd{,} \hlkwc{order} \hlstd{=} \hlnum{5}\hlstd{,} \hlkwc{log_first} \hlstd{=} \hlnum{TRUE}\hlstd{,} \hlkwc{prob} \hlstd{=} \hlnum{0.9}\hlstd{)}
\end{alltt}
\end{kframe}
\end{knitrout}



Table~\ref{tbl:mosquito-ll} compares
the log-likelihood values of the three model versions.
In terms of in-sample model fit,
as measured by the log-likelihood value,
both the negative binomial and the transformation model
perform much better than the Poisson GAMM.
The results suggest slight improvement in the model fit when we relax the
conditional distribution assumption of the negative-binomial GAMM
and follow the distribution-free transformation model approach.

\begin{table}[!ht]
  \centering
  \caption{Log-likelihood values of the fitted Poisson and negative binomial GAMMs
    reproduced from \protect\citet{Juarez_2021} along with the log-likelihood
    of an additive transformation model for count data.}
  \label{tbl:mosquito-ll}

% latex table generated in R 4.1.2 by xtable 1.8-4 package
% Mon Mar  7 15:39:17 2022
\begin{tabular}{lr}
  \toprule
 & Log-likelihood \\ 
  \midrule
Poisson GAMM & -6875.73 \\ 
  Negative binomial GAMM & -4883.26 \\ 
  Additive count transformation model & -4873.07 \\ 
   \bottomrule
\end{tabular}


\end{table}

We will now concentrate on comparing the estimates
from the negative binomial and the count transformation models.
Note that the scales on which the parameters are interpreted
are different in the two models:
While, in the negative binomial model,
the parametric and smooth terms affect
the log of the conditional mean of the outcome,
in the transformation model with ``logit'' link
(i.e., ``expit'' inverse link),
they are interpreted on the log-hazard scale.
Unlike in the example application of Section~\ref{sec:ecoli},
we cannot easily transform the negative binomial parameters
to the probability scale.
Although the magnitudes of the effect estimates
of the two models are not directly comparable,
their directions, significance
and the general patters of the smooths can be contrasted.

Figure~\ref{fig:mosquito-smooth} compares the smooth estimates
of the GAMM from \pkg{gamm4}
and the transformation model from \pkg{tramME}.
Although the within-year time patterns (\texttt{s(Week)})
from the two models are almost identical
(on different scales),
the differences of the smooth estimates of the coverage rate
(\texttt{s(CovRate200)}) are marked.
The general patterns of the smooths are similar,
but the negative binomial GAMM penalizes it more,
which is also reflected in the EDFs:
2.96
and 17.49
for the negative binomial and count transformation models,
respectively.

\begin{figure}[!ht]
  \centering

\begin{knitrout}\small
\definecolor{shadecolor}{rgb}{0.969, 0.969, 0.969}\color{fgcolor}
\includegraphics[width=.9\linewidth]{figure/plot-mosquito-smooth-1} 
\end{knitrout}

\caption{Smooth terms from the negative binomial and transformation models
  of the \emph{A. aegypti} counts.
  The dashed lines and the grey areas denote
  the 95\% confidence intervals}\label{fig:mosquito-smooth}
\end{figure}

Because the parametric and smooth terms
of the two models are defined on different scales,
the magnitudes of the effect estimates are not directly comparable.
As Table~\ref{tbl:mosquito-fe} shows,
the directions of the effects match
and neither model finds evidence that the main effect
of middle income is different from zero.

\begin{table}
  \centering
  \caption{
    Point estimates and 95\% confidence intervals of the parametric
    fixed effects terms from the negative binomial
    and count transformation models
    of the mosquito control data by \protect\citet{Juarez_2021}.
    Note that the scale of the parameters are different
    and the effect sizes are not directly comparable.
  }\label{tbl:mosquito-fe}

% latex table generated in R 4.1.2 by xtable 1.8-4 package
% Mon Mar  7 15:39:23 2022
\begin{tabular}{@{}lrrrr@{}}
  \toprule
  & \multicolumn{2}{c}{Negative binomial} & \multicolumn{2}{c}{Count transformation} \\
 \cmidrule(lr){2-3} \cmidrule(lr){4-5} & $\widehat\beta$ & 95\% CI  & $\widehat\beta$ & 95\% CI  \\
 \midrule
Year = 2018 & $-0.20$ & $-0.34$ ---$-0.06$ & $-0.35$ & $-0.55$ ---$-0.15$ \\ 
  Income = middle & $-0.78$ & $-1.69$ ---$\phantom{-}0.13$ & $-0.83$ & $-2.02$ ---$\phantom{-}0.36$ \\ 
  Placement = out & $2.37$ & $\phantom{-}2.22$ ---$\phantom{-}2.52$ & $3.01$ & $\phantom{-}2.79$ ---$\phantom{-}3.24$ \\ 
  Income = middle \& Placement = out & $0.38$ & $\phantom{-}0.13$ ---$\phantom{-}0.64$ & $0.51$ & $\phantom{-}0.16$ ---$\phantom{-}0.86$ \\ 
   \bottomrule
\end{tabular}


\end{table}

\clearpage

\bibliographystyle{plainnat}
\bibliography{ref}

\clearpage

\begin{knitrout}\small
\definecolor{shadecolor}{rgb}{0.969, 0.969, 0.969}\color{fgcolor}\begin{kframe}
\begin{alltt}
\hlstd{R> }\hlkwd{sessionInfo}\hlstd{()}
\end{alltt}
\begin{verbatim}
R version 4.1.2 (2021-11-01)
Platform: x86_64-pc-linux-gnu (64-bit)
Running under: Ubuntu 20.04.4 LTS

Matrix products: default
BLAS:   /usr/lib/x86_64-linux-gnu/blas/libblas.so.3.9.0
LAPACK: /usr/lib/x86_64-linux-gnu/lapack/liblapack.so.3.9.0

locale:
 [1] LC_CTYPE=en_US.UTF-8       LC_NUMERIC=C              
 [3] LC_TIME=en_GB.UTF-8        LC_COLLATE=en_US.UTF-8    
 [5] LC_MONETARY=en_GB.UTF-8    LC_MESSAGES=en_US.UTF-8   
 [7] LC_PAPER=en_GB.UTF-8       LC_NAME=C                 
 [9] LC_ADDRESS=C               LC_TELEPHONE=C            
[11] LC_MEASUREMENT=en_GB.UTF-8 LC_IDENTIFICATION=C       

attached base packages:
[1] stats     graphics  grDevices utils     datasets  methods   base     

other attached packages:
 [1] gamm4_0.2-6       lme4_1.1-26       Matrix_1.3-3      xtable_1.8-4     
 [5] glmmTMB_1.1.2.3   mgcv_1.8-34       nlme_3.1-152      survival_3.2-13  
 [9] tramME_0.1.2.9000 tram_0.6-1        mlt_1.3-2         basefun_1.1-0    
[13] variables_1.1-1  

loaded via a namespace (and not attached):
 [1] Rcpp_1.0.6          highr_0.8           TMB_1.7.22         
 [4] compiler_4.1.2      nloptr_1.2.2.2      tools_4.1.2        
 [7] boot_1.3-27         statmod_1.4.35      evaluate_0.14      
[10] lattice_0.20-45     polynom_1.4-0       mvtnorm_1.1-1      
[13] xfun_0.23           stringr_1.4.0       BB_2019.10-1       
[16] knitr_1.36          grid_4.1.2          orthopolynom_1.0-5 
[19] multcomp_1.4-17     minqa_1.2.4         TH.data_1.1-0      
[22] alabama_2015.3-1    Formula_1.2-4       magrittr_2.0.1     
[25] codetools_0.2-18    splines_4.1.2       MASS_7.3-54        
[28] numDeriv_2016.8-1.1 quadprog_1.5-8      sandwich_3.0-1     
[31] stringi_1.5.3       coneproj_1.14       zoo_1.8-9          
\end{verbatim}
\end{kframe}
\end{knitrout}

\end{document}
